\documentclass[12pt]{article}
\usepackage[utf8]{inputenc}
\usepackage[T1]{fontenc}
\usepackage{amsmath, amssymb}
\usepackage{graphicx}
\usepackage{geometry}
\usepackage{enumitem}
\usepackage{fancyhdr}
\usepackage{hyperref}

\geometry{a4paper, margin=2cm}

\begin{document}


% Header
\noindent
\begin{tabular}{|p{0.7\textwidth}|p{0.25\textwidth}|}
\hline
\textbf{Teste Automatizado} & \textbf{Versão: A} \\
Curso: Engenharia & Data: 02/12/2025 \\
\hline
\multicolumn{2}{|l|}{Nome: \_\_\_\_\_\_\_\_\_\_\_\_\_\_\_\_\_\_\_\_\_\_\_\_\_\_\_\_\_\_\_\_\_\_\_\_\_\_\_\_\_\_\_\_\_\_\_\_\_\_\_\_\_\_\_\_\_\_\_\_\_\_\_\_\_\_\_} \\
\hline
\end{tabular}

\vspace{0.5cm}

\begin{enumerate}
    \item $\text{Calcule: } \frac{d}{dx}(x^2)$
    \begin{enumerate}[label=\Alph*)]
            \item $2$
            \item $x$
            \item $x^2$
            \item $2x$
        \end{enumerate}
    \vspace{0.3cm}
    \item $\text{Se } A = \begin{pmatrix} 1 & 2 \\ 3 & 4 \end{pmatrix}, \text{ então } A^T \text{ é:}$
    \begin{enumerate}[label=\Alph*)]
            \item $\begin{pmatrix} 1 & 3 \\ 2 & 4 \end{pmatrix}$
            \item $\begin{pmatrix} 1 & 2 \\ 3 & 4 \end{pmatrix}$
            \item $\begin{pmatrix} 1 & 0 \\ 0 & 1 \end{pmatrix}$
            \item $\begin{pmatrix} 4 & 3 \\ 2 & 1 \end{pmatrix}$
        \end{enumerate}
    \vspace{0.3cm}
\end{enumerate}

\vfill

\noindent
\begin{minipage}{0.6\textwidth}
    \textbf{Gabarito:} \\
    \begin{tabular}{|c|c|c|c|c|c|}
        \hline
        Q & A & B & C & D & E \\
        \hline
                1 & & & & & \\ \hline
                2 & & & & & \\ \hline
            \end{tabular}
\end{minipage}
\begin{minipage}{0.35\textwidth}
    \centering
    \includegraphics[width=4cm]{D:/Meu Drive/ger_provas/static/qr_codes/5faca21b-5d97-4b33-9ec5-96c20854280f.png} \\
    \vspace{0.2cm}
    \footnotesize \url{http://localhost:5000/student/5faca21b-5d97-4b33-9ec5-96c20854280f} \\
    Escaneie para enviar o gabarito
\end{minipage}

\clearpage

% Header
\noindent
\begin{tabular}{|p{0.7\textwidth}|p{0.25\textwidth}|}
\hline
\textbf{Teste Automatizado} & \textbf{Versão: B} \\
Curso: Engenharia & Data: 02/12/2025 \\
\hline
\multicolumn{2}{|l|}{Nome: \_\_\_\_\_\_\_\_\_\_\_\_\_\_\_\_\_\_\_\_\_\_\_\_\_\_\_\_\_\_\_\_\_\_\_\_\_\_\_\_\_\_\_\_\_\_\_\_\_\_\_\_\_\_\_\_\_\_\_\_\_\_\_\_\_\_\_} \\
\hline
\end{tabular}

\vspace{0.5cm}

\begin{enumerate}
    \item $\text{Se } A = \begin{pmatrix} 1 & 2 \\ 3 & 4 \end{pmatrix}, \text{ então } A^T \text{ é:}$
    \begin{enumerate}[label=\Alph*)]
            \item $\begin{pmatrix} 1 & 0 \\ 0 & 1 \end{pmatrix}$
            \item $\begin{pmatrix} 4 & 3 \\ 2 & 1 \end{pmatrix}$
            \item $\begin{pmatrix} 1 & 2 \\ 3 & 4 \end{pmatrix}$
            \item $\begin{pmatrix} 1 & 3 \\ 2 & 4 \end{pmatrix}$
        \end{enumerate}
    \vspace{0.3cm}
    \item $\text{Calcule: } \frac{d}{dx}(x^2)$
    \begin{enumerate}[label=\Alph*)]
            \item $x^2$
            \item $x$
            \item $2x$
            \item $2$
        \end{enumerate}
    \vspace{0.3cm}
\end{enumerate}

\vfill

\noindent
\begin{minipage}{0.6\textwidth}
    \textbf{Gabarito:} \\
    \begin{tabular}{|c|c|c|c|c|c|}
        \hline
        Q & A & B & C & D & E \\
        \hline
                1 & & & & & \\ \hline
                2 & & & & & \\ \hline
            \end{tabular}
\end{minipage}
\begin{minipage}{0.35\textwidth}
    \centering
    \includegraphics[width=4cm]{D:/Meu Drive/ger_provas/static/qr_codes/d12f323e-97d4-402d-baf9-eb805029053d.png} \\
    \vspace{0.2cm}
    \footnotesize \url{http://localhost:5000/student/d12f323e-97d4-402d-baf9-eb805029053d} \\
    Escaneie para enviar o gabarito
\end{minipage}

\clearpage

\end{document}